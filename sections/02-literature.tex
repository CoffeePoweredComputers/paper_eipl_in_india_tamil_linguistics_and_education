\begin{table*}[ht]
\centering
\renewcommand{\arraystretch}{1.8}
\begin{tabularx}{\textwidth}{>{\bfseries\raggedright\arraybackslash}X >{\raggedright\arraybackslash}X >{\raggedright\arraybackslash}X >{\raggedright\arraybackslash}X} 
  \hline
\rowcolor{steelblue}
\textcolor{white}{NEP Edition} & \textcolor{white}{\textbf{1968}~\cite{NEP1968}} &
\textcolor{white}{\textbf{1986}~\cite{NEP1986}} &
\textcolor{white}{\textbf{2020}~\cite{NEP2020}} \\ \hline
\rowcolor{palesteelblue}
Hindi-Speaking States & Hindi, English, and a modern Indian language (preferably from Southern India) & Hindi, English, and a modern Indian language (preferably from Southern India) & Three languages of the student's or state's choice (at least two must be native to India) \\
\rowcolor{white}
Non-Hindi Speaking States & Regional language, Hindi, and English & Regional language, Hindi, and English & Three languages of the student's or state's choice (at least two must be native to India) \\
\rowcolor{palesteelblue}
Key Change & First formal introduction of the formula to promote national
integration & Reiterated the 1968 formula without changes & Introduced
flexibility, allowing students to choose the languages, with the condition that
two of the three languages must be Indian \\ \hline
\end{tabularx}
\vspace{0.5em}
\caption{Evolution of the Three-Language Formula in National Education Policies}
\label{tab:three-language-formula}
\end{table*}

To situate the work presented in this paper, we provide an overview of four
relevant sources of literature. First, in
Section~\ref{subsec:indian-higher-education}, we provide a broad overview of the
Indian higher education system with a particular focus on the role that
multilingualism plays in shaping policy and practice. Next, in
Section~\ref{subsec:theories-of-translanguaging},

\subsection{Indian Higher Education \& Language}~\label{subsec:indian-higher-education}

Here, we provide a broad overview of Indian higher education both from the
perspectives of policy and practice. In
Sections~\ref{subsubsec:multilingualism-in-indian-education} and
\ref{subsubsec:language-and-education-policy}, we discuss the multilingual
nature of India and Indian education, respectively. We follow this with an
overview of emperical work on language and Indian higher education in
Section~\ref{subsubsec:emperical-work}.

\subsubsection{Multilingual Landscape}\label{subsubsec:multilingualism-in-indian-education}

% Indian lnguistic landscape
The linguistic landscape of India is incredibly diverse, to such an extent that
it can be difficult to exaclty quantify the number of languages spoken in the
country. India has 22 scheduled languages---meaning they are officially
recognized by the government and have a special status in the constitution---and 
many hundreds of other languages and dialects that are spkoken arcoss its 28
states and 8 union territories~\cite{2011census}.


\subsubsection{Language and Education
Policy}\label{subsubsec:language-and-education-policy}

% NEP and the three language policty
India's National Education Policy (NEP) of 2020~\cite{NEP2020} emphasizes the importance of
of equity and the centrality of language in achieving equitable education.
\begin{quote}
  ``\textit{The aim must be for India to have an education system by 2040 that
  is second to none, with equitable access ot the highest quality educaiton for
  all learners regardless of social or ecenomic background}''(p. 3)
\end{quote}
In previous editions of the NEP, which took places in 1968 and 1980, the
three-language formula has been a key policy in Indian education system---though
it has undergone considerable evolution in it's most recent iteration
(Table~\cite{tab:three-language-formula }). This policy mandates that students
learn three languages where the specific languages differed based primarily on
whether a student lived in a Hindi or non-Hindi speaking state. The 2020 edition
of the NEP, though continuin to emphasize the importance of learning three
languages introuces greater flexibility in which languages fit into each
category and leaves it to the state or region to make these policy decision.

\subsubsection{Emperical Work}\label{subsubsec:emperical-work}


\subsection{Language and Computing Education}

% The rise of natural language programming
Programming, once the domain of structured syntax and rigid semantics, is
increasingly shifting towards natural language interfaces for program
specification~\cite{lau2023ban, prather2025beyond, chen2025empirical,
petrovska2024incorporating}---courtesy of the advanced program synthesis
capabilities of large language models~\cite{austin2021program, jimenezswe}. The
prevelance of English venacular in programming language keywords and
documentation has long been a barrier to entry for non-English
speakers~\cite{guo2018non}. However, the shift towards natural language
prompts---brought about courtesy of multilingual GenAI models, presents both
new opportunities as well as new challenges for non-English speakers. Below, we
highlight some of the key challenges non-English speaking novice programers
have faced historically as well as some of the opprotunities that may arise as
programming moves towards natural language interfaces.

\subsubsection{Potential Challenges for non-English Speakers}



% Role of Syntax and Language
As noted, by \citet{guo2018non}, the vast majority of programming
languages---and almost centainly all widely adopted programming
languages---rely on English keywords. \citet{stefik2013emperical} compared
error rates by novice programmers among a variety of programming languages
(i.e., Quorum, Python, Java, C, Pearl) to that of a language that used random
keywords (i.e., Randomo). They found that languages that used syntax that more
naturally resembled natural language (i.e., Quorum and Python) had lower error
rates and that students had similar error rates for Randomo and languages like
Java. This not only highlights the role that syntax plays but, as hypothesized
by \citet{becker2019parlez}, that non-English speakers may not be able
to share in the affordances of languages such as Quorum and Python. This
hypothesis was, in part, confirmed by the work of \citet{dasgupta2017learning}
which found that Scratch users coding in their localized native language
demonstrate new programming concepts at a faster rate than users from the same
countries using English interfaces, controlling for activity levels and
socialization. 


As noted---somewhat profetically in retreospect given it was published before
widely availbe GenAI models---by \citet{becker2019parlez} as programming
language design moves away from random keywords, to keywords that invoke
natural language, towards fully natural language interface, the difficulty gap
in aquiring that language will become steadily larger and favor the user who is
already familiar with the natural language which the interface is based on. 
\begin{quote}
   ''However, imagine if perfect natural (English) language programming was
   achievable today.''
\end{quote}
Imagine indeed. However, we are fortunate in that large language models are
trained on a wide variety of languages, suggesting that interfacing with these
models in non-English languages may be possible and thus lower the barrier to 
entry for non-English speakers.


\subsubsection{Opportunities for non-English Speakers}

Perhaps the immediate, and most obvious, opportunity is the ability to interact
with programming environments in one's native language~\cite{smith2024explain,
li2024bridging, raihan2024mhumaneval, prather2025breaking}.
As noted by 



\subsubsection{India Specific Opportunities and Challenges}

\citet{bhattacharyya2019indic} characterized a wide variety of challenges relating
to Indic languages and computing. These included, 
\begin{enumerate} 
  \item \textbf{Scale and Diversity}
  Indic languages encompass a vast array of languages and dialects, belonging to multiple linguistic families and written in numerous distinct scripts.
  \item \textbf{Longer Utterances}
    Sentences in Indic languages are often longer and more complex than in English, complicating tasks like parsing and speech recognition.
  \item \textbf{Code Mixing}
    The frequent mixing of multiple languages in a single sentence or conversation is a common challenge in computational linguistics for the region.
  \item \textbf{Resource Scarcity}
    Many Indic languages lack sufficient annotated datasets for building robust NLP and speech tools.
  %\item \textbf{Absence of basic speech and NLP tools}
  %  Foundational tools like morphology analyzers and speech recognition systems are either unavailable or lack accuracy for most Indic languages.
  \item \textbf{Absence of linguistics knowledge}
    A limited understanding of the linguistic structure of many regional languages hinders the development of computational models.
  \item \textbf{Script complexity and non-standard input mechanisms}
    The diversity of scripts and their associated vowel and consonant combinations make input systems slower and less intuitive.
  \item \textbf{Non-standard transliteration}
    Roman transliteration of Indic languages lacks standardization, leading to multiple ways of representing the same word.
  %\item \textbf{Non-standard storage}
  %  Variations in how characters are encoded and stored pose issues in data sharing and tool interoperability.
  %\item \textbf{Man-made Problems}
  %  Government-imposed standard keyboards and inadequate funding often stifle innovation and efficiency in linguistic computing.
  %\item \textbf{Some challenging language phenomena}
  %  Features like free word order, agglutination, and context-dependent pronunciation introduce additional computational hurdles.
\end{enumerate}
Though issues such as \textit{}

\citet{} found that 
\citet{smith2024explain} had similar findings with Tamil translations of
correct English descriptions of code having far less success than 10 other of
hte most commonly spoken languages in India



\subsection{Pedagogy and Linguistic Diversity}

\subsubsection{Theories of Translanguaging}

% Definitions, terms, etc

% Theories of translanguaging

% Translanguaging work in india




