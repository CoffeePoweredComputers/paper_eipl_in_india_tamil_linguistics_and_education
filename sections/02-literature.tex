\begin{table*}[ht]
\centering
\renewcommand{\arraystretch}{1.8}
\begin{tabularx}{\textwidth}{>{\bfseries\raggedright\arraybackslash}p{3.5cm} >{\raggedright\arraybackslash}X >{\raggedright\arraybackslash}X}
  \hline
\rowcolor{steelblue}
\textcolor{white}{NEP Edition} & \textcolor{white}{\textbf{1968 \& 1986}~\cite{NEP1968,NEP1986}} &
\textcolor{white}{\textbf{2020}~\cite{NEP2020}} \\ \hline
\rowcolor{palesteelblue}
Hindi-Speaking States & Hindi, English, and a modern Indian language (preferably from Southern India) & Three languages of the student's or state's choice (at least two must be native to India) \\
\rowcolor{white}
Non-Hindi Speaking States & Regional language, Hindi, and English & Three languages of the student's or state's choice (at least two must be native to India) \\
\rowcolor{palesteelblue}
Key Change & 1968: First formal introduction of the formula to promote national integration\newline\newline 1986: Reiterated the 1968 formula without changes & Introduced flexibility, allowing students to choose the languages, with the condition that two of the three languages must be Indian \\ \hline
\end{tabularx}
\vspace{0.5em}
\caption{Evolution of the Three-Language Formula in National Education Policies}
\label{tab:three-language-formula}
\end{table*}

To situate the work presented in this paper, we provide an overview of three
key areas of literature. First, in Section~\ref{subsec:indian-higher-education},
we examine the Indian higher education system, focusing on how multilingualism
shapes both educational policy and classroom practice. Second, in
Section~\ref{subsec:language-and-computing-education}, we explore the evolving
relationship between language and computing education, particularly the shift
towards natural language programming interfaces and its implications for
non-English speakers in India. Finally, in
Section~\ref{subsec:pedagogy-and-linguistic-diversity}, we review theoretical
frameworks and pedagogical approaches for leveraging linguistic diversity in
educational contexts, including translanguaging, culturally relevant pedagogy,
and critical language awareness.

\subsection{Indian Higher Education \& Language}~\label{subsec:indian-higher-education}

Here, we provide a broad overview of Indian higher education both from the
perspectives of policy and practice. In
Sections~\ref{subsubsec:multilingualism-in-indian-education} and
\ref{subsubsec:language-and-education-policy}, we discuss the multilingual
nature of India and Indian education, respectively. We follow this with an
overview of empirical work on language and Indian higher education in
Section~\ref{subsubsec:empirical-work}.

\subsubsection{Multilingual Landscape}\label{subsubsec:multilingualism-in-indian-education}

% Indian linguistic landscape
The linguistic landscape of India is incredibly diverse, to such an extent that
it can be difficult to exactly quantify the number of languages spoken in the
country. India has 22 scheduled languages---meaning they are officially
recognized by the government and have a special status in the constitution---and 
many hundreds of other languages and dialects that are spoken across its 28
states and 8 union territories~\cite{2011census}. \textcolor{red}{TODO: Add 
some more of Mohanty and alls work on characterizing the landscape.}


\subsubsection{Language and Education
Policy}\label{subsubsec:language-and-education-policy}

% NEP and the three language policty
India's National Education Policy (NEP) of 2020~\cite{NEP2020} emphasizes the importance of
of equity and the centrality of language in achieving equitable education.
\begin{quote}
  ``\textit{The aim must be for India to have an education system by 2040 that
  is second to none, with equitable access to the highest quality education for
  all learners regardless of social or economic background}''(p. 3)
\end{quote}
In previous editions of the NEP, which took places in 1968 and 1980, the
three-language formula has been a key policy in Indian education system---though
it has undergone considerable evolution in its most recent iteration
(Table~\ref{tab:three-language-formula}). This policy mandates that students
learn three languages where the specific languages differed based primarily on
whether a student lived in a Hindi or non-Hindi speaking state. The 2020 edition
of the NEP, though continuing to emphasize the importance of learning three
languages introduces greater flexibility in which languages fit into each
category and leaves it to the state or region to make these policy decision.

\textcolor{red}{
  TODO: Add some context regarding praise and criticism of the NEP 2020.
  \blindtext[1]
}

\subsubsection{Empirical Work}\label{subsubsec:empirical-work}

\textcolor{red}{
  TODO: Fill in he K-12 education stuff I've found.
  \blindtext[1]
}

\textcolor{red}{
  TODO: Fill in Gerald's work on Tamil in higher education.
  \blindtext[1]
}


\subsection{Language and Computing Education}\label{subsec:language-and-computing-education}

% The rise of natural language programming
Programming, once the domain of structured syntax and rigid semantics, is
increasingly shifting towards natural language interfaces for program
specification~\cite{lau2023ban, prather2025beyond, chen2025empirical,
petrovska2024incorporating}---courtesy of the advanced program synthesis
capabilities of large language models~\cite{austin2021program, jimenezswe}. The
prevalence of English vernacular in programming language keywords and
documentation has long been a barrier to entry for non-English
speakers~\cite{guo2018non}. However, the shift towards natural language
prompts---brought about courtesy of multilingual GenAI models, presents both
new opportunities as well as new challenges for non-English speakers. Below, we
highlight some of the key challenges non-English speaking novice programmers
have faced historically as well as some of the opprotunities that may arise as
programming moves towards natural language interfaces.

\subsubsection{Potential Challenges for non-English Speakers}



% Role of Syntax and Language
As noted, by \citet{guo2018non}, the vast majority of programming
languages---and almost certainly all widely adopted programming
languages---rely on English keywords. \citet{stefik2013empirical} compared
error rates by novice programmers among a variety of programming languages
(i.e., Quorum, Python, Java, C, Pearl) to that of a language that used random
keywords (i.e., Randomo). They found that languages that used syntax that more
naturally resembled natural language (i.e., Quorum and Python) had lower error
rates and that students had similar error rates for Randomo and languages like
Java. This not only highlights the role that syntax plays but, as hypothesized
by \citet{becker2019parlez}, that non-English speakers may not be able
to share in the affordances of languages such as Quorum and Python. This
hypothesis was, in part, confirmed by the work of \citet{dasgupta2017learning}
which found that Scratch users coding in their localized native language
demonstrate new programming concepts at a faster rate than users from the same
countries using English interfaces, controlling for activity levels and
socialization. 


As noted---somewhat prophetically in retrospect given it was published before
widely avail be GenAI models---by \citet{becker2019parlez} as programming
language design moves away from random keywords, to keywords that invoke
natural language, towards fully natural language interface, the difficulty gap
in acquiring that language will become steadily larger and favor the user who is
already familiar with the natural language which the interface is based on. 
\begin{quote}
   ''However, imagine if perfect natural (English) language programming was
   achievable today.''
\end{quote}
Imagine indeed. However, we are fortunate in that large language models are
trained on a wide variety of languages, suggesting that interfacing with these
models in non-English languages may be possible and thus lower the barrier to 
entry for non-English speakers.


\subsubsection{Opportunities for non-English Speakers}

Perhaps the immediate, and most obvious, opportunity is the ability to interact
with programming environments in one's native language~\cite{smith2024explain, prather2025breaking}.
As noted by 



\subsubsection{India Specific Opportunities and Challenges}

% Opportunities


% Challenges
However, even prior to the advent of GenAI models, it is important to understand
that Natural Language Processing (NLP) technologies have long faced challenges
in effectively supporting Indic languages. \citet{bhattacharyya2019indic}
characterized a wide variety of challenges relating to Indic languages and
computing. These included, 
\begin{enumerate} 
  \item \textbf{Scale and Diversity}
  Indic languages encompass a vast array of languages and dialects, belonging to multiple linguistic families and written in numerous distinct scripts.
  \item \textbf{Longer Utterances}
    Sentences in Indic languages are often longer and more complex than in English, complicating tasks like parsing and speech recognition.
  \item \textbf{Code Mixing}
    The frequent mixing of multiple languages in a single sentence or conversation is a common challenge in computational linguistics for the region.
  \item \textbf{Resource Scarcity}
    Many Indic languages lack sufficient annotated datasets for building robust NLP and speech tools.
  \item \textbf{Absence of linguistics knowledge}
    A limited understanding of the linguistic structure of many regional languages hinders the development of computational models.
  \item \textbf{Script complexity and non-standard input mechanisms}
    The diversity of scripts and their associated vowel and consonant combinations make input systems slower and less intuitive.
  \item \textbf{Non-standard transliteration}
    Roman transliteration of Indic languages lacks standardization, leading to multiple ways of representing the same word.
\end{enumerate}
Recent work indicates some of these issues are far from resolved.
\citet{jordan2024need} found that problem statements for programming problems
generated in Tamil by GPT-3.5 were often non-sensical and hypothesized this may
be due to the non-latin alphabet or insufficient training data.
\citet{smith2024explain} had similar findings with Tamil translations of correct
English descriptions of code having far less success in generating correct
code---using GPT-4o---than 10 other of he most commonly spoken languages in
India. These findings highlight that, though the multilingual nature of the
current state of the art GenAI models presents new opportunities for non-English
speakers, these opportunities are not necessarily equitably distributed across
all languages. This is of particular concern in the Indian context where
minority and tribal languages are already at a systemic disadvantage in the
education system~\cite{mohanty2017language} and are almost certainly
underrepresented in the training data of these models.



\subsection{Pedagogy and Linguistic Diversity}\label{subsec:pedagogy-and-linguistic-diversity}

To understand how instructors navigate multilingual computing classrooms,
we examine pedagogical frameworks that reconceptualize linguistic diversity
as a resource rather than a barrier in educational settings
(Sections~\ref{subsubsec:theories-of-translanguaging}--\ref{subsubsec:critical-language-awareness}). These theoretical perspectives—including
translanguaging, code-switching, culturally relevant pedagogy, and critical
language awareness—provide lenses for interpreting instructor practices and
their potential to create more equitable learning environments in India's
linguistically diverse computing education landscape.

\subsubsection{Theories of Translanguaging}\label{subsubsec:theories-of-translanguaging}

The term \textit{translanguaging} was first coined by Cen Williams in their
work at the University of Wales, Bangor~\cite{}. It
was derived from the Welsh term \textit{trawsieithu}---which means
``to cross languages''---and was used in the context of Welsh/English classrooms. 
The concept gained broader recognition and 

% Translanguaging work in programming
Though analysis of translanguaging in programming education is limited, there is
one notable exception. \citet{tai2024transprogramming} introduced a theory of
\textit{transprogramming}. They expand on the idea of translanguaging to
describe the process by which teachers and students navigate multilingual,
multimodal, and programming language resources to accomplish programming tasks
and develop mental models of of computational concepts.

% Translanguaging work in india --- Highlight the need for this work and what
% makes it unique.
\textcolor{red}{
  \blindtext[1]
}


\subsubsection{Code-switching and Code-meshing}\label{subsubsec:code-switching-and-code-meshing}

% Highlight the difference between translanguaging (full linguistic repetoir),
% code-switching (swtiching languages), and code-meshing (mixing languages).
\textcolor{red}{
  \blindtext[1]
}

\subsubsection{Funds of Knowledge and Culturally Relevant Pedagogy}\label{subsubsec:funds-of-knowledge}

% Define funds of knowledge and CRC
\textcolor{red}{
  \blindtext[1]
}

% Disucuss empirical findings for why these things are a good idea
\textcolor{red}{
  \blindtext[1]
}

\subsubsection{Critical Language Awareness}\label{subsubsec:critical-language-awareness}

% Some things that are mroe at the meta level. Why language matters, power, etc.
% The system we create, reinforce, and perpetuate via education.
\textcolor{red}{
  \blindtext[1]
}


