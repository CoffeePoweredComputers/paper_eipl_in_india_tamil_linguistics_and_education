\citet{bhattacharyya2019indic} characterized the challenges of NLP in Indic
languages as,
\begin{enumerate} \item \textbf{Scale and Diversity}
  Indic languages encompass a vast array of languages and dialects, belonging to multiple linguistic families and written in numerous distinct scripts.
  \item \textbf{Longer Utterances}
    Sentences in Indic languages are often longer and more complex than in English, complicating tasks like parsing and speech recognition.
  \item \textbf{Code Mixing}
    The frequent mixing of multiple languages in a single sentence or conversation is a common challenge in computational linguistics for the region.
  \item \textbf{Resource Scarcity}
    Many Indic languages lack sufficient annotated datasets for building robust NLP and speech tools.
  \item \textbf{Absence of basic speech and NLP tools}
    Foundational tools like morphology analyzers and speech recognition systems are either unavailable or lack accuracy for most Indic languages.
  \item \textbf{Absence of linguistics knowledge}
    A limited understanding of the linguistic structure of many regional languages hinders the development of computational models.
  \item \textbf{Script complexity and non-standard input mechanisms}
    The diversity of scripts and their associated vowel and consonant combinations make input systems slower and less intuitive.
  \item \textbf{Non-standard transliteration}
    Roman transliteration of Indic languages lacks standardization, leading to multiple ways of representing the same word.
  \item \textbf{Non-standard storage}
    Variations in how characters are encoded and stored pose issues in data sharing and tool interoperability.
  \item \textbf{Man-made Problems}
    Government-imposed standard keyboards and inadequate funding often stifle innovation and efficiency in linguistic computing.
  \item \textbf{Some challenging language phenomena}
    Features like free word order, agglutination, and context-dependent pronunciation introduce additional computational hurdles.
\end{enumerate}





In \citet{jordan2024need}, they found that amoung the languages under their
consideration, English, Spanish, Vietnamese, and Tamil, the worst performing
language was Tamil, generating incorrect solutions and non-sensiable
translations. Though the authors did not systematically investigate the the
nature of the poor performance or methods of improvement, they did note several
possible reasons:
\begin{enumerate}
  \item 
  \item Compared to English and Spanish it was likely a lower resourced language
    in the training data.
\end{enumerate}
The first of these two possibilities is, perhaps, the most interesting. In
Tamil, there are two core dialects which differ significantly: literary Tamil
(sen-Tamil) and colloquial (kodun-Tamil)~\cite{}. Additionally, given Tamil is
spoken not just in and around Tamil Nadu, but also in Sri Lanka, Malaysia, and
Singapore, there are a number of regional dialects~\cite{}. 
