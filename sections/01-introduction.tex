% GenAI is shifting the use of natural language to the forefront of computing education.
% There are historic inequities related to access that may be exacurbated by this.
% India represents one of the largest technical education systems in the world.
% They are also one of the most linguistically diverse. 
% There is a gap in research related to the imapct of lingustic diversity in post-secondary education
% Relevance of the Gap: This may create larger inequities than ever before
% This was probably work investigating before but is a MUST now.
% Our work will aim to provide a phenomenographical studyof instructor perceptions and experiences dealiing with lingustic diversity in computing classrooms in India across a variety of instiutions and contexts.

The rise of generative AI (GenAI) tools is fundamentally transforming education
and how it is practiced across various disciplines~\cite{prather2025beyond,
bond2024meta} and leading institutions to integrate it into all elements of
their educational models~\cite{southworth2023developing}. In computing
education, there has been a focus on how the use of these tools may cause
writing code from scratch to take a backseat to This shift brings with it both
opportunities---namely the ability to enable student to engage inmore
personally meaningful and complex projects~\cite{porter2024learn,
bottcher2025concepts}--and challenges related to the importance of teaching
students to clearly articulate requests and requirements in natural
language~\cite{becker2023programming, denny2024prompt, smith2024prompting,
reeves2024prompts}. Given the existing prevelance of English as the dominant
medium of instruction in computing education~\cite{guo2018non}, the multilingual
nature of the state-of-the-art GenAI models presents both opportunities to
overcome existing inequities as well as challenges that may exacerbate
them~\cite{yong2023prompting, smith2024explain, prather2025breaking}.

Recent work by \citet{jacob2022examining} and \citet{cheung2025systematic}
notes there is little work on the experiences of multilingual students in
computing education---and those that do appear to focus on the North American
contex. Additionally, \citet{cheung2025systematic} notes that the majority of
work on translanguaging is focused on science and math education, with only a
small fraction of work (6 studies) focused on engineering disciplines. As
computing education continues to shift towards natural language tools for 
the dearth of work informing how to best support multilingual students in
computing education becomes more pressing.

India presents a unique context for examing these dynamics. The country
simultaneously represents one of the world's largest technical education
systems~\cite{goel2017echnical} and one of the most linguistically diverse, with
22 official languages and over 700 documentated languages---and many more
dialects that vary in differences of degree---that are spoken across its
states~\cite{2011census}. This linguistic diversity, in the computing education
context, embodies what \citet{mohanty2017language} refered to as a ``double
divide'' between English and dominant regional languages, as well as between
dominant and minority regional languages. The context is further complicated by
the paradox of the prevelance of English in programming languages---along with
the resources to learn them---and the emphasis on mother-tongue instruction that
both India's National Education Policy (NEP) 2020 and the UNESCO's longstanding 
guidance on education policy have advocated for~\cite{unesco1953vernacular, NEP2020}.

Though this gap should always have been addressed, the rise and prevelance of
GenAI tools in computing education brings with it a new found urgency to 
address the issue of linguistic diversity in computing education. The primary
means of interacting with GenAI models is through natural language. As such,
this could deepen existing inquities between those who are fluent in
English---or other dominant languages of instruction in India---and those who
are not. Further more, there is the overarching issue of the performance of
these models in languages that are not as well represented in their training
data, which could further exacerbate existing inequities.


To address these gaps, this paper presents a phenomenographics study of
computing instructors' perceptions of and experiences with linguistic diversity
in Indian higher education. Drawing on semi-structured interviews with XX
instructors across XX institutions in XX states, we analyze how educators
navigate the realities of linguistic diversity in their classrooms.
Our analysis employs theoretical frameworks from translanguaging pedagogy,
culturally relevant education, and critical language awareness to interpret
instructor practices and their potential for creating equitable learning
environments. This study aims to advance the theoretical understanding of
linguistic diversity in computing and technology education while offering
practical insights for more equitable pedagogy in an era where natural
language mediates the act of computing itself.
